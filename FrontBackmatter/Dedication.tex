
%*******************************************************
% Dedication
%*******************************************************
\cleardoublepage
\thispagestyle{empty}
%\phantomsection 
\pdfbookmark[1]{Dedication}{Dedication}

\vspace*{3cm}

\begin{quotation}

   \openquote In the good old days, theorizing was like sailing between islands of
   experimental evidence. And, if the trip was not in the vicinity of the
   shoreline (which was strongly recommended for safety reasons) sailors where
   continuously looking forward, hoping to see land --- the sooner the better.
   
   Nowadays, some theoretical physicists (let us call them sailors) [have]
   found a way to survive and navigate in the open sea of pure theoretical
   constructions. Instead of the horizon, they look at stars, which tell them
   exactly where they are. Sailors are aware of the fact that the stars will
   never tell them where the new land is, but they may tell them their position
   on the globe. 

   Theoreticians become sailors simply bacause they just like it. Young people,
   seduced by capitans forming crews to go to a Nuevo El Dorando \textelp{} soon
   realize that they will spend all their life at sea. Those who do not like
   sailing desert the voyage, but for the true potential sailors the sea become
   their passion. They will probably tell the alluring and frightening truth to
   their students --- and the proper people will join their ranks.~\closequote

   \begin{flushright}
      ---  Andrei Losev
   \end{flushright}

\end{quotation}

\medskip

%\begin{center}
%    Dedicated to the loving memory of Rudolf Miede. \\ \smallskip
%    1939\,--\,2005
%\end{center}










