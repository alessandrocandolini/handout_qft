%*******************************************************
% Prefazione 
%*******************************************************

\chapter*{Preface}
\markboth{\spacedlowsmallcaps{Preface}}{\spacedlowsmallcaps{Preface}} 
%\mtcaddchapter[\tocEntry{\protect\vspace1em}]
%\mtcaddchapter[\numberline{}\tocEntry{Preface}]

%\addcontentsline{toc}{\protect\vspace{\beforebibskip}}
		   %\addtocontents{toc}{\protect\vspace{\cftbeforepartskip}}
		   \addtocontents{toc}{\protect\vspace{\beforebibskip}}
\addcontentsline{toc}{chapter}{\tocEntry{\numberline{}Preface}}

Relativistic quantum theory of fields (QFT) is an extremely fashinating topic,
full of far-reaching implications in several areas of modern Physics and
which reveals a rich mathematical structure behind it. It is the language
currently 
used to develop a major importance part of latest achievements in theoretical
physics in several areas, ranging from high energy particle physics to 
statistical mechanics, solid state physics, cosmology,
etc. It is a source of
intriguing new physical questions and a furnace where to develop new challanging
mathematical ideas.
 

%It is difficult to give an outlook of the major field of reaserch 
QFT was originally developed in order to approach
quantum mechanics in a way consistent with the requirements of Einstein's special relativity. 
Currently, it offers a comprehensive framework to
deal with
elementary excitations above the ground state of physical systems having an
infinite number of degrees of freedom.
 
QFT is the theoretical framework currently employed to
describe all foundamental interactions but
gravitation in high-energy physics --- in the Standard Model of 
particle physics,
strong interactions are described by Quantum Chromodynamics (QCD), which is 
an unbroken non-abelian $SU(3)$ color Yang-Mills gauge QFT, while the electroweak
interactions are described by a non-abelian $SU(2)
\times U(1)$ Yang-Mills gauge QFT spontaneously broken via the Higgs mechanism.
Furthermore, QFT has
let to a deeper understanding of the singular properties of a wide class of phase
transitions at the critical point in statistical field theory
and has allowed the development of field-theoretical descriptions
of the statistical properties of
certain geometrical models (\eg, self-avoiding random walks which are of major
importance, for instance,  
in polymer physics).

The level of mathematical rigour will be suitable to an i

Sometimes we will run into 
lengthly calculations.
Hopefully, this will be helpful.
Learing QFT be doing it, this is the basic learning method behind this handout.
Otherwise, it could become harder and dangerous to acquire familiarity
with the tools and strategies of QFT without getting lost into the formalism.



%\begin{signature}
%   \Large{\calligra\myName}
%\end{signature}

\mySignature{\textsw{\myLocation, \myTime}}{\Large{\calligra\myName}}

%\begin{flushright}
%    %\begin{tabular}{c}
%        \Large{\calligra\myName} 
%    %\end{tabular}
%\end{flushright}

