
%*******************************************************
% Chapter 2
%*******************************************************


\myChapter{Functional methods for scalar quantum fields} 


%This chapter is devoted to the application of path integrals and functional
%techniques to the quantum theory of relativistic scalar fields.

The general strategy behind functional methods in quantum field theory is
similar to that a statistical mechanics:
\begin{itemize}
   \item build up the key object of the theory, which is the \emph{generating}
      functional
      \begin{dmath}
	 Z[J]  = \pathint{ \exp{ - \frac{i}{\hslash} S[\varphi]}}{\varphi}
      \end{dmath}
      in which the dynamics of the fields is \emph{encoded};
   \item extract (\ie, \emph{decode}) the physical quantities from the
      generating 
      functional.
\end{itemize}
$Z(J)$ plays the role of the partition function in statistical mechanics.
In fact, physicists are used to employ the same symbol $Z$ to denote
both of them.
Primary interest is on physical quantities like
\begin{inparaenum}[a)]
   \item scattering cross-sections,
   \item decay rates of unstable particles,
   \item magnetic moments, etc.
\end{inparaenum}
However, the theoretical calculations of these quantities usually proceed
through  the following
two steps:
\begin{itemize}
   \item first, more abstract quantities, \ie, the $n$-point correlation
      function, are computed from the generating functional,
   \item then the $S$ matrix which determined scattering cross-sections and
      decay rates is related to correlation functions.
\end{itemize}





\section{Path integral quantization of scalar fields}

The Lagrangian density $\Lagrangian\free$ of the non-interacting real neutral massive scalar field
$\varphi(x)$ in the $(d+1)$-dimensional Minkowski space is%
\footnote{Our conventions are summarized in \cref{chp:lorenz}.}
\begin{dmath}[label={L0scalar}]
   \Lagrangian\free \left( \varphi(x), \partial_{\mu}\varphi(x) \right) = \frac{1}{2} \left( \partial_{\mu} \varphi(x) \right) \left(
\partial^{\mu} \varphi(x) \right) - \frac{1}{2} m^{2} \varphi^{2}(x) 
\end{dmath}.
It is a \emph{function} of the field and its first-order space-time
derivatives at one space-time point $x$.
The parameter $m$ will later be identified with the bare mass of the bosons.

The corresponding action functional $S[\varphi(x)]$ is defined as usual as the space-time integral of the
Lagrangian density, \ie, 
\begin{equation}\label{eq:S}
S[\varphi(x)] = \int \Lagrangian \d[4]{x} = 
\int \left( \frac{1}{2} \left( \partial_{\mu} \varphi(x) \right) \left(
\partial^{\mu} \varphi(x) \right) - \frac{1}{2}m^{2} \varphi^{2}(x) \right)
\d[4]{x} \eqspace .
\end{equation}
$S$ is a real-valued \emph{functional} of the field configuration $\varphi$.



\subsection{Path integral quantization of interacting scalar field}
\subsection{Review of canonical quantization of scalar fields}

In order to make some comparisons between path integral and operator approaches
to the quantum theory of scalar fields, we briefly sketch a summary of canonical
quantization of scalar fields.


\section{Perturbation expansion of Green functions of self-interacting scalar fields}
\section{Feynman diagrams}

\begin{dmath}
   \feyn{fs f fl flu f fs}
   %\feyn{fs f flA fluV f fs}
\end{dmath}

\section{Troubles with perturbation theory}

\section{Symmetry factors of Feynman diagrams for scalar fields}

