
%*******************************************************
% Chapter 4
%*******************************************************


\myChapter[Funzioni contenenti funzioni esponenziali]{Primitive di funzioni che contengono funzioni esponenziali e
logaritmiche} 
%\minitoc\mtcskip
\begin{Exercise}
Calcolare
\begin{displaymath}
\int \frac{\uexp^{x}}{1+\uexp^{x}} \udiff{x} \eqspace ,\qquad
\int \frac{1}{1+\uexp^{x}} \udiff{x} \eqspace .
\end{displaymath}
\end{Exercise}
\begin{Solution}
Il primo integrale \`e immediato. Riconosciamo infatti al numeratore la derivata
della funzione al denominatore, quindi usando la solita
\begin{displaymath}
\int \frac{1}{f(x)}f^{\prime}(x) \udiff{x} = \log \Modul{f(x)} + c \eqspace, \qquad
c\in\R \eqspace, 
\end{displaymath}
con $f(x) = 1+ \uexp^{x}$ e quindi $f^{\prime}(x)  = \uexp^{x}$ si ha
\begin{displaymath}
\int \frac{\uexp^{x}}{1+\uexp^{x}} \udiff{x} = \log \Modul {1+\uexp^{x}} + c
\eqspace, \qquad c \in \R \eqspace .
\end{displaymath}
\par 
Lo stesso argomento non funziona per il secondo integrale, perch\'e al
numeratore c'\`e $1$ invece di $\uexp^{x}$ (e 1 non \`e la derivata del
denominatore).  Un modo standard di procedere \`e quello di usare il metodo di
sostituzione, si pone ad esempio 
\begin{displaymath}
t = \uexp^{x} \eqspace, 
\end{displaymath}
quindi 
\begin{displaymath}
x = \log t \eqspace ,
\end{displaymath}
 e la regola mnemonica
\begin{displaymath}
\udiff{x} = \frac{1}{t} \udiff{t} \eqspace, 
\end{displaymath}
per cui
\begin{displaymath}
\int \frac{1}{1+\uexp^{x}} \udiff{x} = \int \frac{1}{1+t} \frac{1}{t} \udiff{t}
\eqspace .
\end{displaymath}
Quest'ultimo integrale si pu\`o svolgere notando che (procedimento di
scomposizione in fratti semplici) 
\begin{displaymath}
\frac{1}{t (1+t)} =  \frac{1}{t} - \frac{1}{1+t} 
\eqspace ,
\end{displaymath}
 da cui 
\begin{eqnarray*}
\int \frac{1}{t(1+t)} \udiff{t} &=& \left( \int \frac{1}{t}
\udiff{t} - \int \frac{1}{t+1} \udiff{t} \right) \\
&=& \log \Modul{t} - \log \Modul{t+1} + c \eqspace, \qquad c\in\R
\eqspace .
\end{eqnarray*}
Si ha pertanto
\begin{displaymath}
\int \frac{1}{1+\uexp^{x}} \udiff{x} = \log \Modul{\uexp^{x}} - \log \Modul{1+
\uexp^{x}} + c = x - \log \left( 1+ \uexp^{x} \right) + c\eqspace, \qquad c
\in\R \eqspace .
\end{displaymath}
C'\`e un modo pi\`u veloce di calcolare questo integrale senza usare il metodo
di sostituzione, basta riscrivere la funzione integranda in questo modo:
\begin{eqnarray*}
\frac{1}{1+\uexp^{x}} &=& \frac{1 +\uexp^{x} - \uexp^{x}}{1+\uexp^{x} } \\
&=& \frac{1+ \uexp^{x}}{1+ \uexp^{x}} - \frac{\uexp^{x}}{1+\uexp^{x}} \\
&=& 1 - \frac{\uexp^{x}}{1+\uexp^{x}} \eqspace, 
\end{eqnarray*}
da cui in maniera diretta 
\begin{displaymath}
\int\frac{1}{1+\uexp^{x}} \udiff{x} = \int \left( 1 -
\frac{\uexp^{x}}{1+\uexp^{x}} \right) \udiff{x} = x - \log \Modul{1+\uexp^{x}} +
c\eqspace, \qquad c\in\R\eqspace .
\end{displaymath}
\end{Solution}
\begin{Exercise}
Calcolare
\begin{displaymath}
\int \frac{1}{\uexp^{x} + \uexp^{-x}} \udiff{x}  \eqspace, \qquad 
\int \frac{\uexp^{x}}{1+\uexp^{2x}} \udiff{x}
\eqspace .
\end{displaymath}
\end{Exercise}
\begin{Solution}
Anche in questo caso il metodo standard potrebbe essere quello di procedere per
sostituzione
\begin{displaymath}
 t = \uexp^{x} \eqspace, \qquad x = \log t \eqspace, \qquad \udiff{x} =
\frac{1}{t} \udiff{t} \eqspace ,
\end{displaymath}
quindi 
\begin{eqnarray*}
\int \frac{1}{\uexp^{x} + \uexp^{-x}}\udiff{x} &=& \int \frac{1}{t + \frac{1}{t}}
\frac{1}{t} \udiff{t} \\
&=& \int \frac{t}{t^{2} + 1 } \frac{1}{t} \udiff{t} \\
&=& \int \frac{1}{1+t^{2}} \udiff{t} \\
&=& \arctan t + c \eqspace , \qquad c\in\R \eqspace, \\
&=& \arctan \uexp^{x} + \eqspace, \qquad c \in \R \eqspace .
\end{eqnarray*}
Allo stesso risultato si poteva giungere anche senza usare il metodo di
sostituzione, infatti riscrivendo la funzione integranda come
\begin{displaymath}
\frac{1}{\uexp^{x} + \uexp^{-x}} = \frac{1}{\uexp^{-x} \left( \uexp^{2x} + 1
\right)} = \frac{\uexp^{x}}{1+ \left( \uexp^{x} \right)^{2}} \eqspace, 
\end{displaymath}
e ricordando
\begin{displaymath}
\int \frac{1}{1+\left[ f(x)\right]^{2}} f^{\prime}(x) \udiff{x} = \arctan f(x) +
c\eqspace, \qquad c \in\R \eqspace, 
\end{displaymath}
si ha direttamente 
\begin{eqnarray*}
\int \frac{1}{\uexp^{x} + \uexp^{-x}} \udiff{x}  &=& 
\int\frac{\uexp^{x}}{1+ \left( \uexp^{x} \right)^{2}} \udiff{x} \\
&=& \arctan \uexp^{x} + c\eqspace , \qquad c\in\R \eqspace .
\end{eqnarray*}
Questo risponde anche alla seconda parte dell'esercizio.
\end{Solution}
%\begin{homework}
%Calcolare
%\begin{displaymath}
%\int \frac{\uexp^{x}-1}{\uexp^{x}+1} \udiff{x} \eqspace .
%\end{displaymath}
%\end{homework}
\begin{Exercise}
Ricordiamo che non \`e possibile esprimere
\begin{displaymath}
\int \uexp^{-x^{2}} \udiff{x}
\end{displaymath}
in termini di funzioni elementari.%
\footnotemark
%\begin{comment}
La funzione $\uexp^{-x^{2}}$ \`e la funzione
``a campana'' di Gauss, o Gaussiana, \`e molto importante in parecchi contesti
anche applicativi, \eg,
fisica, ingegneria, statistica, etc. Le sue primitive sono scrivibili in termini di
funzioni speciali, in particolare usando la \emph{error function}:
\begin{displaymath}
\int \uexp^{-x^{2}} \udiff{x} = \frac{\sqrt{\pi}}{2} \erf x + c \eqspace ,
\qquad c\in\R \eqspace.
\end{displaymath}
%\end{comment}
Il grafico delle funzioni $\uexp^{-x^{2}}$ e $\erf x$  sono in
\figurename~\ref{fig:plot_gauss}.
\par
Non \`e invece difficile calcolare (farlo per esercizio) i seguenti
\begin{displaymath}
\int x \uexp^{-x^{2}} \udiff{x}  \eqspace, \qquad 
\int x^{3} \uexp^{-x^{2}} \udiff{x} \eqspace .
\end{displaymath}
\end{Exercise}
\footnotetext{Questo fatto \`e spesso menzionato, pi\`u raramente ne viene riportata la
dimostrazione.
Al riguardo, si pu\`o consultare, \eg, il capitolo~8 di G.~Boros, V.~H.~Moll,
\emph{Irresistible Integrals. Symbolics, Analysis and Experiments in the
Evaluation of Integrals}, Cambridge Univ. Press, Cambridge (2004), e le
referenze all'interno.}
\begin{figure}
\centering
%\subfigure[][]{\includegraphics{plot_gauss1}}
%\subfigure[][]{\includegraphics{plot_gauss2}}
\caption{Plot delle funzioni $\uexp^{-x^{2}}$ (la curva a campana di Gauss) e
$\erf x $ (la error function) per valori reali di $x$.\label{fig:plot_gauss}}
\end{figure}
\begin{Solution}
Il primo integrale \`e immediato, essendo infatti
\begin{displaymath}
\deriv{}{x} \uexp^{-x^{2}} = -2x \uexp^{-x^{2}} \eqspace, 
\end{displaymath}
si trova che
\begin{displaymath}
\int x \uexp^{-x^{2}} \udiff{x} = -\frac{1}{2} \uexp^{-x^{2}} + c \eqspace ,
\qquad c\in\R \eqspace .
\end{displaymath}
Per quanto riguarda il secondo integrale, si proceda per parti nel modo
seguente:
\begin{eqnarray}
\int x^{3} \uexp^{-x^{2}} \udiff{x} &=& \int \underbrace{x^{2}}_{g}
\underbrace{x \uexp^{-x^{2}}}_{f^{\prime}} \udiff{x} \label{eq:parti}\\
&=& \underbrace{x^{2}}_{g} \underbrace{ \left( - \frac{1}{2} \uexp^{ -x^{2}}
\right)}_{f} - \int \underbrace{\left( -\frac{1}{2} \uexp^{-x^{2}} \right)}_{f}
\underbrace{2x}_{g'} \udiff{x} \nonumber \\
&=& -\frac{1}{2} x^{2} \uexp^{-x^{2}}  + \int  x \uexp^{-x^{2}} \udiff{x}
\nonumber \\
&=& - \frac{1}{2} \uexp^{-x^{2}} \left( x^{2} +1 \right) + c \eqspace ,\qquad
c\in\R \eqspace .\nonumber 
\end{eqnarray}
\end{Solution}
\begin{homework}
Avrei potuto scegliere
\begin{displaymath}
f^{\prime}(x) = \uexp^{-x^{2}}  \eqspace, \qquad g(x) = x^{3} \eqspace, 
\end{displaymath}
oppure
\begin{displaymath}
g(x)  = \uexp^{-x^{2}}  \eqspace, \qquad f^{\prime}(x)= x^{3} \eqspace, 
\end{displaymath}
nella Eq.~\eqref{eq:parti}?
\end{homework}
\begin{Exercise}
Calcolare
\begin{displaymath}
\int \log x \udiff{x} \eqspace, \qquad 
%\int \log \frac{x-1}{x+1} \udiff{x}\eqspace , \qquad 
\int \frac{1}{x\log^{4} x } \udiff{x} \eqspace .
\end{displaymath}
\end{Exercise}
\begin{Solution}
Procediamo per parti:
\begin{eqnarray*}
\int \log x \udiff{x} &=& \int \underbrace{1}_{f^{\prime}} \underbrace{\log x }_{g}
\udiff{x} \\
&=& \underbrace{x}_{f} \underbrace{\log x }_{g} - \int \underbrace{\frac{1}{x}
}_{g'} \underbrace{x \vphantom{\frac{1}{x}}}_{f} \udiff{x} \\
&=& x \log x - \int 1 \udiff{x} \\
&=& x \log x - x + c \eqspace, \qquad c \in\R \eqspace .
\end{eqnarray*}
Il secondo integrale \`e immediato, usando
\begin{displaymath}
\int  \left[ f(x)\right] ^{\alpha } f^{\prime}(x) \udiff{x} = \frac{\left[
f(x)\right]^{\alpha+1}}{\alpha +1 } +  c \eqspace , \qquad c\in\R \eqspace, 
\end{displaymath}
valida per ogni $\alpha \in\R$, $\alpha \neq -1$, si ha
\begin{displaymath}
\int \frac{1}{x} \log^{-4} x \udiff{x} = \frac{\log^{-4+1} x}{-4+1} + c 
= - \frac{1}{3\log^{3} x } + c \eqspace , \qquad c\in\R \eqspace. 
\end{displaymath}
\end{Solution}


