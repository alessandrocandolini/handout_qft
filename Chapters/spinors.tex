
%*******************************************************
% Appendix C
%*******************************************************
\myChapter{Spinors}
%\minitoc\mtcskip

\lettrine{S}{pinors} are fashinating objects.
They were first invented by mathematician Cartan to classify Lie algebras.
In mathematics they are related to 
\begin{itemize}
   \item classification of Lie algebras;
   \item quaternions;
   \item representations of groups, in particular: the rotation group, and the
      Lorentz group;
   \item Clifford algebras;
\end{itemize}
In physics, spinor fields describe fermions.
The link between physics and mathematics is related 

This chapter  provide a review and summary of the
Lorentz group, its algebra and its representations, with focus on the spinor
representations.
This section is mostly based on \textcite{Dreiner.Haber.ea:2010}.

\section{The Lorentz group}

This section is mostly based on \textcite{Barone:2004}.

\subsection{Notations and conventions}

The metric tensor of the flat Minkowsky space (\ie, special relativity) is
taken to be with signature $(+,-,-,-)$, \ie,
\begin{dmath}
   g^{\mu\nu} \hiderel{=} g_{\mu\nu} = \diag( 1, -1, -1,-1)
\end{dmath}.

The homogeneus Lorentz transformations are \emph{linear}, \emph{homogeneus}
transformations
\begin{dmath}
   @x\prime^{\mu}@ = @\Lambda^{\mu}_{\nu}@ @x^{\nu}@
\end{dmath}
that leave invariant the interval $@ x^{\nu} @@x_{\nu}@$.
   



