%*******************************************************
% Appendix E
%*******************************************************


\myChapter{Saddle-point method and WKB}

The 
stepeest descent method (also referred to as saddle point method) is a powerful tool to handle the problem of
evaluating  the
  asymptotic behavior (or at least its leading term) of parametric integrals of a certain form.
For example, it might be useful in evaluating the asymptotic behavior of functions
 for which a suitable integral representation is available.
An introductory discussion of these topics is given, \eg, by
\cite[\S~11]{king}, \cite[\S6]{ablowitz}. See also Zinn-Justin.
\par
Let us begin with a special case of real integrals over one real variable (this
is known also as Laplace's method) and then generalize to contour integrals on
a  complex domain. Generalization to an arbitrary number of variables is not
covered here, even if such generalizations are often needed in the toolkit of
every physicists.%
\footnote{Stepeest  descent method  generalizes also to
functional
integrations and path integrals, this is very important in dealing with
non-perturbative corrections in quantum mechanics and quantum field
theory.}
%we restrict to integrals with respect to a
%\emph{real} variable 
%(Laplace method). This is enough for example to develop the Stirling's
%approximation of $\Gamma(x)$ for real values of $x$ as $x\rightarrow+\infty$ and thus deriving the
%Stirling formula of $n!$.
%Then we will move to the more general case involving contour integrals in the
%complex
%plane. This is the case if you seek for example the asymptotic behavior of  
%$\Gamma(z)$
%for complex values of $z$ as $\modul{z} \rightarrow +\infty$.
%\par
\section{Real case (alias, Laplace method)}
\section{Complex case}
Consider the complex integral  along the real line:
\begin{dmath}
   I(\lambda) = \Int{ g(t)  \exp{ \lambda f(t)}}{t,\Omega} 
\end{dmath},
where $\Omega\subseteq \R$ is an interval of the real line,
$f:\Omega\rightarrow\R$ a \emph{real}-valued function on $\Omega$,
$g:\Omega\rightarrow\C$a  \emph{complex}-valued function on $\Omega$ and
$\lambda$ a \emph{positive} real parameter.
We will study the leading term in the asymptotic expansion of integrals of such
form as $\lambda \rightarrow+\infty$. This is more properly referred to as
Laplace method, then we will move to the more general case of contour integrals
in the complex plane.
\par
\begin{figure}
   \centering
%\includegraphics{saddle}
\caption{Saddle point}
\end{figure}




